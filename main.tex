\documentclass{foi}
\usepackage{lipsum}
\usepackage[utf8]{inputenc}
\usepackage{float}

\lstset{basicstyle=\ttfamily,
  showstringspaces=false,
  commentstyle=\color{red},
  keywordstyle=\color{blue},
}

\vrstaRada{\zavrsni}
\title{Izrada aplikacije za pronalazak termina sastanaka}

\author{Leo Ćavar}
\spolStudenta{\musko}
\mentor{Marko Mijač}
\spolMentora{\musko}
\godina{2024}
\mjesec{Rujan}
\date{2024}
\status{redoviti}
\indeks{0016153823}
\smjer{Informacijski i poslovni sustavi}
\titulaProfesora{Doc. Dr. sc.}

\sazetak{U ovom završnom radu se obrađuje korištenje .NET tehnologija za izradu programa za dogovaranje sastanka. Kroz rad se obrađuje kratko o ASP. NET Core i ASP. NET Web API tehnologijama, kao i koncepte API-ja i HTTP metoda, Google API-ja, uključujući Google Calendar API, autentifikaciju i autorizaciju korisnika pomoću OAuth 2.0 protokola, te primjere zahtjeva za dohvaćanje i upravljanje kalendarskim događajima kroz implementaciju .NET web aplikacije 
}

\kljucneRijeci{API; ASP. NET Core; .NET; C\#; ASP. NET; ; OAuth 2.0; Google Calendar; Google API;}

\begin{document}

\maketitle

\tableofcontents

\pagestyle{plain}
\chapter{Uvod}
U ovom radu ćemo se usredotočiti na izradu programa za dogovaranje sastanaka kroz upotrebu Google API-ja, koji nam omogućava interakciju s iznimno popularnim Google kalendarom. Realizirat ćemo program koristeći ASP.NET Core i Razor stranice za izradu front-end sučelja te upravljanje podacima, dok će ASP.NET Web API služiti za primanje HTTP zahtjeva i komuniciranje s Google servisima za manipuliranje događajima.

\chapter{Metode i tehnike rada}
Za pisanje teksta i formatiranje ovog rada koristio se LaTeX unutar programa Visual Studio Code. Za izradu praktičnog dijela korišten je Visual Studio 2022 i JetBrains Rider.

\chapter{API}
Kada korisnik koristi softver kao klijent, često koristi nekakvo softversko sučelje za interakciju s softverom, ali kada je potrebno da jedan softver koristi dijelove drugog softvera tada koristimo vrstu sučelja za programiranje aplikacija (engl. \textit{Application Programming Interface}) ili skraćeno API.\cite{biehl2015api}
Ta interakcija se najčešće bazira na tome da klijent šalje HTTP zahtjev serveru na određenu lokaciju i dobivaju se nazad podaci.
API zahtjev se sastoji od nekoliko dijelova \cite{altexsoft}
\begin{itemize}
    \item Operacija koja se izvršava (primjer. \textit{GET, POST})
    \item Autentifikacijski parametri
    \item Odredište - URL API završne točke (engl. \textit{endpoint})
\end{itemize}
Poziv može sadržavati i druge parametre ali ovo su tri osnovna koja će se uvijek koristiti.
\section{HTTP zahtjevi}
Kada klijent šalje zahtjev poslužitelju mora specifirati u zahtjevu koju metodu želi izvršiti, imena metoda se odnose na ono što želimo postići sa zahtjevom. \cite{Maurya2021} 
\begin{itemize}
    \item \textbf{GET} metoda se koristi za dohvaćanje podataka
    \item \textbf{POST} - slanje i dodavanje podataka
    \item \textbf{PUT} - Ažuriranje podataka
    \item \textbf{DELETE} - brisanje resursa 
\end{itemize}

\section{RESTFul API}
RESTFul API je vrsta API-ja koja prati REST \textit(eng. {representational state transfer}) principe dizajna, može biti u bilo kojem jeziku i može koristiti bilo koju vrstu podataka \cite{ibm_rest_api}.
Iako najčešće koristi HTTP protokol on nije nužno vezan za njega.\cite{Microsoft2023}
\begin{itemize}
    \item \textbf{Jedinstveno sučelje} - API dizajn mora biti konzistentan i predvidljiv, s pristupom resursima putem standardnih HTTP metoda kao što su GET, POST, PUT i DELETE.
    
    \item \textbf{Razdvajanje klijenta i servera} - Klijent i server su neovisni, gdje server ne čuva informacije o stanju klijenta između zahtjeva, a klijent nema direktan pristup serverovim podatcima.
    
    \item \textbf{Bezustanje (engl. \textit{Stateless})} - Svaki zahtjev od klijenta prema serveru mora sadržavati sve potrebne informacije za obradu, bez potrebe za čuvanjem stanja na serveru.
    
    \item \textbf{Keširanje} - Resursi se mogu keširati kako bi se smanjilo opterećenje servera i omogućilo ponovnu upotrebu već preuzetih podataka.
    
    \item \textbf{Sustav slojeva} - Slojevita arhitektura omogućuje umetanje posrednika između klijenta i servera, dodajući funkcionalnosti poput keširanja ili sigurnosnih provjera.
    
    \item \textbf{Kôd na zahtjev (opcionalno)} - Klijent može preuzeti i izvršiti kod od servera radi proširenja funkcionalnosti aplikacije. 
\end{itemize}

\chapter{ASP.NET Core}
\section{ASP.NET MVC}
ASP.NET MVC je framework koji se bazira na MVC (engl. \textit{Model-View-Controller}) arhitekturi, izgrađen je na .NET platformi i koristi se za izradu web aplikacija \cite{Tyler2024}.
Kao što ime glasi, MVC arhitektura se sastoji od modela, pregleda (eng. \textit{View}) i kontrolera (eng. \textit{Controller}). Ovaj oblik dizajna prati prvi princip SOLID metoda, razdvajanja odgovornosti (eng. \textit{Seperation of concerns}).
MVC omogućava ponovnu upotrebljivost, i zbog podjele na 3 glavne komponente olakšava održavanje koda. \cite{GeeksforGeeks2024}
\begin{figure}[H]
    \centering
    \includegraphics[width=0.6\textwidth]{slike/MVC_project.jpeg}
    \caption{Prikaz MVC u ASP.NET MVC projektu (Izvor: autor)}
    \label{fig:mvc_projekt}
\end{figure}

\subsection{Model}
Unutar konteksta ASP.NET MVC projekta, model predstavlja C\# klasu koja sadrži svojstva za spremanje podataka kojima upravljamo. Model je neovisan o korisničkom sučelju ali često će postojati pregled (engl. textit{View}) koji odgovara za prikaz i upravljanje modelom.
Model također može sadržavati poslovnu logiku za upravljanje podacima, iako to nije učestala praksa i često se ta uloga daje servisima.
\subsection{View}
Pregledi (eng. \textit{Views}) se koriste za prikazivanje podataka i korisničku interakciju. ASP.NET MVC koristi Razor stranice, sa ekstenzijom \textit{cshtml}. Razor stranice omogućavaju pisanje C\# koda unutar HTML datoteka koji služi za interaktiranje sa HTML oznakama za generiranje web sadržaja \cite{Smith2022}. Najčešće će svaki pregled imati svoj kontroler koji je odgovoran za rad s pregledima.

\subsection{Controllers}
Kontroler u ASP.NET Core MVC arhitekturi služi kao posrednik između modela i prikaza. On obrađuje korisničke zahtjeve, upravlja podacima iz modela, i odlučuje koji će prikaz biti vraćen korisniku. Kontroleri su ključni dio MVC uzorka jer povezuju poslovnu logiku s korisničkim sučeljem.
Kontroler je klasa koja obično nasljeđuje baznu klasu \texttt{Controller} i sadrži metode koje se nazivaju akcije (engl. \textit{actions}). Svaka akcija odgovara određenom korisničkom zahtjevu i vraća rezultat, kao što je prikaz (\textit{ViewResult}), JSON podaci (\textit{JsonResult}), ili redirekcija (\textit{RedirectResult}).
Akcije također možemo opisati kao metode koje se povezivaju kad unesemo određeni URL. \cite{Walther2022}

\section{ASP.NET Web API}
ASP.NET Core nam omogućava da kreiramo web API s upotrebom kontrolera koji su usredočeni na resurse i primanje HTTP zahtjeva. Prednost kreiranja zasebnog Web API projekta je da ga može koristiti više različitih vrsta klijenta.\cite{ASPNet2023}
U ASP.NET Core možemo imati dva pristupa kreiranja API-ja:
\begin{itemize}
    \item \textbf{API bazirani na kontrolerima (engl. \textit{controller-based APIs})}: U ovom pristupu, kontroleri (engl. \textit{controllers}) su klase koje nasljeđuju \texttt{ControllerBase} klasu. Ove klase koriste se za definiranje API krajnjih točaka (engl. \textit{endpoints}). Metode unutar kontrolera mapiraju se na određene HTTP zahtjeve (engl. \textit{HTTP requests}, npr. GET, POST) i vraćaju odgovore u obliku JSON-a, XML-a, ili drugih formata.

    \item \textbf{Minimalni API-ji (engl. \textit{minimal APIs})}: Ovaj pristup omogućava definiranje krajnjih točaka pomoću lambda izraza (engl. \textit{lambda expressions}) ili metoda, bez potrebe za punom klasom kontrolera. Minimalni API-ji su dizajnirani za jednostavne i brze implementacije, gdje se fokusira na definiranje krajnjih točaka uz minimalno opterećenje infrastrukture.
\end{itemize}
Za potrebe ovog rada koristit će se API bazirani na kontrolerima zbog bolje organizacije koda u cjeline te zbog jednostavnijeg rukovanja ulaznim podacima pomoću atributa \textit{[FromBody]} i \textit{[FromQuery]} 

\chapter{Praktični rad: Meeting Planner}
\section{Uvod u projekt}
Prije izrade projekta trebaju se definirati funckionalnosti projekta. Cilj aplikacije je dohvaćanje svih dostupnih termina koje korisnik ima u svome Google kalendaru te sinkronizacija u jedan ispis koji će organizatoru sastanka prikazati kada su dostupni svi planirani korisnici i odabrana dvorana u kojoj će se sastanak održati. 
Na bazi toga, definiramo funckionalnosti.
\begin{itemize}
    \item Prijava korisnika: Korisnik se može prijaviti u sustav koristeći svoje Google podatke.
    \item Prikaz kalendara jednog korisnika (organizator): Korisniku se prikazuje njegov osobni kalendar s pregledom svih zakazanih sastanaka u obliku tablice.
    \item Upis termina u kalendare (kalendar organizatora i kalendar korisnika): Nakon pronalaska slobodnog termina, sustav će omogućiti organizatoru upis sastanka u svoj kalendar kao i njihove kalendare.
    \item Pronalazak slobodnih termina: Sustav pretražuje slobodne termine za sastanak uzimajući u obzir slobodno vrijeme sudionika, odabrane dane za sastanak i dostupne lokacije.
\end{itemize}
Za omogućavanje integracije s Google kalendarom koristiti ćemo Google Calendar API. RESTFul API s kojim se može interaktirati pomoću HTTP poziva.
\section{Google Workspace}
Google Workspace nam omogućava platformu za integraciju Google Workspace alata kao što su Maps, Calendar ili Sheets. Za ovaj projekt su nam potrebni Google Workspace APIs da integriramo kalendar s našim rješenjem.
Da možemo koristiti te servise moramo prvo stvoriti projekt u Google Cloud konzoli te omogućiti Google Calendar API za naš projekt.
\begin{figure}[H]
    \centering
    \includegraphics[width=0.9\textwidth]{slike/google_console.png}
    \caption{Google Cloud console (Izvor: autor)}
    \label{fig:google_console}
\end{figure}
\section{Stvaranje projekta}
Za stvaranje projekta prvo je potrebno stvoriti prazno rješenje (engl. \textit{Solution}), te kreirati zasebni WEB API i MVC projekt.
\begin{figure}[H]
    \centering
    \includegraphics[width=0.6\textwidth]{slike/struktura.png}
    \caption{Rješenje MeetingPlanner (Izvor: autor)}
    \label{fig:struktura}

\end{figure}
Za dodatnu konfiguraciju potrebno je preuzeti json datoteku koja sadrži podatke koji našoj aplikaciji omogućuju pozivanje google servisa. 
\begin{itemize}
    \item GoogleId (Identifikator klijenta): jedinstveni identifikator dodijeljen aplikaciji od strane Googlea. Identificira vašu aplikaciju na Googleovim poslužiteljima prilikom slanja zahtjeva, kao što su autentifikacija korisnika ili pristup API-jima.
    \item GoogleSecret (Tajni ključ): povjerljivi ključ povezan s vašom Google aplikacijom. Koristi se zajedno s GoogleId za autentifikaciju vaše aplikacije prema Googleu.
\end{itemize}
\newpage
\section{Autentifikacija}
Kako bi korisnik aplikacije mogao pristupiti vanjskim uslugama on mora biti prijavljen, korisnik će se prijavljivati putem svojeg Google računa koristeći OAuth 2.0 protokol. Prvo je potrebno konfigurati autentifikaciju projekta u \textit{Program.cs} datoteci.
\begin{figure}[H]
    \centering
    \includegraphics[width=0.7\textwidth]{slike/auth.png}
    \caption{Postavljanje autentifikacije (Izvor: autor)}
    \label{fig:autentifikacija_programcs}
\end{figure} Kod konfigurira autentifikaciju postavljanjem kolačića za prijavu i koristeći Google za autentifikaciju korisnika, uključujući konfiguraciju za Google kalendar, pohranu OAuth tokena, i određivanje putanje za povratak korisnika nakon prijave.
Kada je postavljena konfiguracija kreiramo login stranicu, s obzirom da koristimo Google autentifikaciju kreiramo element koji poziva akciju \textit{LoginWithGoogle}.
\begin{figure}[H]
    \centering
    \includegraphics[width=0.9\textwidth]{slike/loginwithgoogle.png}
    \caption{Metoda LoginWithGoogle (Izvor: autor)}
    \label{fig:LoginWithGoogle}

\end{figure}
Metoda \textit{LoginWithGoogle} inicijalizira objekt \textit{AuthenticationProperties} s postavkom \textit{RedirectUri} koja određuje putanju na metodu \textit{Callback} nakon prijave putem Googlea. Metoda \texttt{Challenge} pokreće izazov za autentifikaciju koristeći Google kao pružatelja, preusmjeravajući korisnika na Google za prijavu ako nije već prijavljen. Nakon uspješne prijave, korisnik će biti vraćen na metodu \texttt{Callback}.


\chapter{Zaključak}
.NET okruženje za skriptiranje u GNU/Linux naredbenom retku pruža nove mogućnosti za razvoj i automatizaciju. Kroz rad je demonstriran veći broj Bash i .NET skripti i demonstrirana je integracija .NET alata i Bash ljuske koristeći .NET alat dotnet-shell. Kroz primjere su prikazane prednosti i nedostatci oba sustava. .NET poboljšava interoperabilnost time što se skripte mogu direktno koristiti na svim operacijskim sustavima koji imaju instalirano .NET okruženje. Također, .NET pruža dodatne funkcionalnosti preko svojih biblioteka i NuGet paketa što omogućuje jednostavno razvijanje kompleksnijih skripti čime se još više mogu poboljšati radni procesi. Prednost Bash skripti je što rade na svim GNU/Linux distribucijama koje koriste Bash ljusku bez potrebe za dodatnim instalacijama i zahtjevaju manje resursa od .NET skripti čime su lakše za sustave. Da zaključim, sinergija .NET okruženja i Bash skriptnog jezika u GNU/Linux operacijskom sustavu omogućuje moćno okruženje za projekte automatizacije iz razloga što se može koristiti najbolje od oba jezika pri pisanju skripti, .NET elementi za kompleksne unaprijed pripremljene funkcionalnosti, a Bash za GNU/Linux specifične radnje ukoliko ima potrebe za njima. 

\printbibliography[title=Popis literature]
\addcontentsline{toc}{chapter}{Popis literature}

\listoffigures
\addcontentsline{toc}{chapter}{Popis slika}

\listoftables
\addcontentsline{toc}{chapter}{Popis tablica}

\end{document}
